\documentclass{article}
\usepackage{amsmath}
\usepackage{amssymb}
\usepackage{bm}
\usepackage[parfill]{parskip}
\usepackage{color}
\usepackage{hyperref}
\usepackage[left = 3.5cm, right = 3.5cm, top = 3.0cm, bottom = 4.5cm]{geometry}

\title{Model description}
\author{}
\date{}

\begin{document}

\maketitle

For each bird, counts of \emph{Macrorhabdus ornithogaster} using the
five diagnostic tests were taken on selected days. Let $y_{ijk}$ be
the number of \emph{M.\ ornithogaster} observed for the $i$th bird, on
the $j$th day its samples were taken, using the $k$th test. 

The observed counts exhibit two types of correlation that must be
modelled when it comes to statistical analysis. First, counts from the
same bird and the same day but using different diagnostics tests are
correlated, because both are related to the bird's unknown shedding
rate; if the bird is shedding many \emph{M.\ ornithogaster} organisms
then large numbers are likely to be detected using all diagnostic
tests. Second, counts from different samples taken soon after one
another, but on different days, may be correlated due to the bird's
unknown shedding rate varying slowly over time; a bird shedding many
\emph{M.\ ornithogaster} organisms on a particular day may still have
a high shedding rate the following day.

Our model accounts for both forms of correlation by fitting a
state-space model, where the state process represents a latent
variable related to the unknown shedding rates of the birds, which we
hereafter refer to as a shedding measure. We fit a negative binomial
distribution to each observed count, which is a distribution commonly
used for count data with high variance like ours. The observed count
$y_{ijk}$ has the expectation
\begin{equation}
\lambda_{ijk} = \exp\{\beta_{0k} +  s_{ij}\},
\end{equation}
where $s_{ij}$ is a latent variable related to the shedding measure of
the $i$th bird on the $j$th day its samples were taken, and
$\beta_{0k}$ controls the relationship between this shedding measure
and the expected number of organisms detected using the $k$th
method. A diagnostic test with a large $\beta_{0k}$ is capable of
detecing many organisms, and may therefore be more effective at
diagnosing an \emph{M.\ ornithogaster} infection.

The latent shedding measure of a bird may vary over time due to both
treatment and random day-to-day fluctuations caused by other
processes, such as bird health and immune response. Our model allows
the shedding measure to change over time differently during treatment
and non-treatment periods, for example to accommodate a decrease in
shedding during treatment but a potential increase upon its
cessation. We use
\begin{equation}
  s_{ij} = \alpha_1 t_j + u_j
\end{equation}
for measurements taken during the treatment period, where $t_j$ is
the number of days since the beginning of treatment when the $j$th day
the bird's samples were taken, and
\begin{equation}
  s_{ij} =  \alpha_1 t_e + \alpha_2 (t_j - t_e) + u_j
\end{equation}
for measurements following the cessation of treatment, where $t_e$ is
the total number of treatment days. The parameters $\alpha_1$ and
$\alpha_2$ control the per-day change in the bird's shedding measure
during treatment and following cessation of treatment,
respectively. Finally, $u_j$ accounts for changes in shedding due to
other factors, for example bird health, which we model using a
Gaussian process to account for correlation between two counts from
samples from the same bird taken close together in time.

Comparing $\beta_{0k}$ parameters between different diagnostic tests
allows us to determine if some tests tend to detect larger numbers of
organisms than others. Moreover, our model allows calculation of the
probability of successfully diagnosing a \emph{M.\ ornithogaster}
infection given a particular shedding measure and diagnostic test,
achieved by calculating the probability of a nonzero count under the
fitted negative binomial distribution. To further assess the
diagnostic performance of the tests, we calculated the probability of
successful diagnosis using the different tests, based on the estimated
shedding measures at end-of-treatment and follow up of the birds that
remained infected throughout the experiment.
\end{document}